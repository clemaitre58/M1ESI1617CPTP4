\documentclass[12pt]{tdtp}
\usepackage{tabularx,colortbl}
\usepackage{multirow}
\usepackage{listings}
\lstset{
	language=VHDL,
basicstyle=\tiny\ttfamily}
\definecolor{light-gray}{gray}{0.96}
\definecolor{pageheading-gray}{gray}{0.2}
\definecolor{dark-gray}{gray}{0.45}
\definecolor{dark-green}{rgb}{0.245,0.121,0.0}

\newcommand{\auteur}{Cedric Lemaitre}
\newcommand{\couriel}{c.lemaitre58@gmail.com}
\newcommand{\promo}{M1 ESI}
\newcommand{\annee}{2016-2017}
\newcommand{\matiere}{Composant programmable}

\newcommand{\tdtp}{TP 4}
\renewcommand{\sujet}{VHDL Design}


\begin{document}
\titre
Nous proposons dans ce TP des exercices préparatoires nécessaire à la programmation de la carte Nexys 3\footnote{\url{https://reference.digilentinc.com/_media/nexys:nexys3:nexys3_rm.pdf}}.\\
\textit {NB : tous les exemples doivent faire l'objet de tests avec des \textbf{test-benchs} et à l'aide l'outil de simulation}


%%%%%%%%%%%%
\Exo

 À partir de la documentation de la carte Nexys 3, déterminer le design permettant d'afficher une valeur numérique entre 0 et F sur un des afficheurs 7 segments.
%%%%%%%%%%%%
\Exo

Réalisez un design permettant d'afficher sur un afficheur 7 segments la valleur d'un compteur prenant des valeurs entre 0 et F.

%%%%%%%%%%%
\Exo 

Réaliser un compteur en base 10 permettant d'afficher des valeurs entre 0 et 9999.\\
\textit{Pour plus de clareté, créez une structure de code modulaire}

%%%%%%%%%%
\end{document}
